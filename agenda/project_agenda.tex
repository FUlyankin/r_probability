%!TEX TS-program = xelatex
\documentclass[12pt, a4paper, oneside]{article}

\usepackage{amsmath,amsfonts,amssymb,amsthm,mathtools}  % пакеты для математики

\usepackage[utf8]{inputenc} % задание utf8 кодировки исходного tex файла
\usepackage[british,russian]{babel} % выбор языка для документа

\usepackage{fontspec}         % пакет для подгрузки шрифтов
\setmainfont{Helvetica}   % задаёт основной шрифт документа

% why do we need \newfontfamily:
% http://tex.stackexchange.com/questions/91507/
\newfontfamily{\cyrillicfonttt}{Helvetica}
\newfontfamily{\cyrillicfont}{Helvetica}
\newfontfamily{\cyrillicfontsf}{Helvetica}

\usepackage{unicode-math}     % пакет для установки математического шрифта
\setmathfont{Neo Euler}      % шрифт для математики
% \setmathfont[math-style=ISO]{Asana Math}
% Можно делать смену начертания с помощью разных стилей

% Конкретный символ из конкретного шрифта
% \setmathfont[range=\int]{Neo Euler}

%%%%%%%%%% Работа с картинками %%%%%%%%%
\usepackage{graphicx}                  % Для вставки рисунков
\usepackage{graphics}
\graphicspath{{images/}{pictures/}}    % можно указать папки с картинками
\usepackage{wrapfig}                   % Обтекание рисунков и таблиц текстом

%%%%%%%%%%%%%%%%%%%%%%%% Графики и рисование %%%%%%%%%%%%%%%%%%%%%%%%%%%%%%%%%
\usepackage{tikz, pgfplots}  % язык для рисования графики из latex'a

%%%%%%%%%% Гиперссылки %%%%%%%%%%
\usepackage{xcolor}              % разные цвета

\usepackage{hyperref}
\hypersetup{
	unicode=true,           % позволяет использовать юникодные символы
	colorlinks=true,       	% true - цветные ссылки, false - ссылки в рамках
	urlcolor=blue,          % цвет ссылки на url
	linkcolor=black,          % внутренние ссылки
	citecolor=green,        % на библиографию
	pdfnewwindow=true,      % при щелчке в pdf на ссылку откроется новый pdf
	breaklinks              % если ссылка не умещается в одну строку, разбивать ли ее на две части?
}


\usepackage{todonotes} % для вставки в документ заметок о том, что осталось сделать
% \todo{Здесь надо коэффициенты исправить}
% \missingfigure{Здесь будет Последний день Помпеи}
% \listoftodos --- печатает все поставленные \todo'шки

\usepackage{enumitem} % дополнительные плюшки для списков
%  например \begin{enumerate}[resume] позволяет продолжить нумерацию в новом списке

\usepackage[paper=a4paper, top=20mm, bottom=15mm,left=20mm,right=15mm]{geometry}
\usepackage{indentfirst}       % установка отступа в первом абзаце главы

\usepackage{setspace}
\setstretch{1.15}  % Межстрочный интервал
\setlength{\parskip}{4mm}   % Расстояние между абзацами
% Разные длины в латехе https://en.wikibooks.org/wiki/LaTeX/Lengths


\usepackage{xcolor} % Enabling mixing colors and color's call by 'svgnames'

\definecolor{MyColor1}{rgb}{0.2,0.4,0.6} %mix personal color
\newcommand{\textb}{\color{Black} \usefont{OT1}{lmss}{m}{n}}
\newcommand{\blue}{\color{MyColor1} \usefont{OT1}{lmss}{m}{n}}
\newcommand{\blueb}{\color{MyColor1} \usefont{OT1}{lmss}{b}{n}}
\newcommand{\red}{\color{LightCoral} \usefont{OT1}{lmss}{m}{n}}
\newcommand{\green}{\color{Turquoise} \usefont{OT1}{lmss}{m}{n}}

\usepackage{titlesec}
\usepackage{sectsty}
%%%%%%%%%%%%%%%%%%%%%%%%
%set section/subsections HEADINGS font and color
\sectionfont{\color{MyColor1}}  % sets colour of sections
\subsectionfont{\color{MyColor1}}  % sets colour of sections

%set section enumerator to arabic number (see footnotes markings alternatives)
\renewcommand\thesection{\arabic{section}.} %define sections numbering
\renewcommand\thesubsection{\thesection\arabic{subsection}} %subsec.num.

%define new section style
\newcommand{\mysection}{
	\titleformat{\section} [runin] {\usefont{OT1}{lmss}{b}{n}\color{MyColor1}} 
	{\thesection} {3pt} {} } 


%	CAPTIONS
\usepackage{caption}
\usepackage{subcaption}
%%%%%%%%%%%%%%%%%%%%%%%%
\captionsetup[figure]{labelfont={color=Turquoise}}

\usepackage[normalem]{ulem}  % для зачекивания текста

\pagestyle{empty}

\usepackage{float}

\begin{document}

\section*{Анализ открытых источников данных (aka Mass Research)} 

\todo[inline]{Курс рекомендуется студентам 3 курса. Все материалы, используемые в курсе можно будет найти на страничке курса на Github: \url{https://github.com/FUlyankin/massResearch_houses}}

\subsection*{Что будем делать}

Мы будем делать большой проект.  Вместе. Делать его будем на примере недвижимости. Скачаем данные, предобработаем их, обогатим, отработаем все те знания, которые получили на эконометрике. Проверим несколько гипотез, которые придут к нам в голову, поищем аномалии, попробуем прогнозировать цены и многое другое! В финале нашего проекта мы сделаем интерактивную веб-страничку и, если найдём что-то интересное, небольшую заметку на habr и medium. Весь код будем сохранять в общем репозитории на github в своих ветках. Самые крутые решения будем вливать в общую ветку. 

У меня есть примерный план, ко которому пойдёт наше исследование, но при этом нет чёткого сценария. Для каждой пары я готовлюсь к нескольким вариантам развития событий и поощряю любую вашу инициативу. Что конкретно мы стараемся искать в данных выбираете вы. Цепляйтесь за любые даже самые мимолётные идеи и озвучивайте их, насколько бы глупыми они вам не казались. 

В примерном плане исследования те темы, о которых я хотел бы поговорить. Вы можете сбивать меня и направлять в любую сторону. Даже в сторону разной математической подноготной моделей, если вам это вдруг стало интересно.  

Понятное дело, что мы не успеем сделать всё, но мы попробуем выжать из нашего большого датасета максимум. 


\subsection*{ Примерный план исследования}

\begin{enumerate}
	\item  Накидываем план исследования. Обсуждаем какие гипотезы про недвижимость нам хотелось бы проверить. Регистрируемся на github, присоединяемся к репозиторию, заводим свою ветку, учимся комитить и разрешать конфликты. 
		
	\item Начинаем собирать данные. Пишем на python парсер для CIAN. Набрасываем костяк и и основной цикл для сбора. Отправляемся домой дописывать. 
	
	\item Нас всех забанило. Учимся не злить сервер и модернизируем свои парсеры. Отправляемся домой собирать данные. 
	
	\item Строим первые визуализации. Смотрим разную описательную статистику и снова обсуждаем гипотезы. Думаем как обогатить данные для их проверки. Смотрим на открытые данные и разные API.  Отправляемся домой обогащать.
	
	\item Смотрим кто как обогатил, забираем лучшие решения в мастер. Подключаемся к google maps и качаем географическую информацию про окрестности квартир. Смотрим на Selenium, обогащаем данные поисковыми запросами. Дома доделываем свои итоговые датасеты. 
	
	\item Приводим в порядок git, который вы скорее всего сломали.  Предобработка и варка фичей. Основы работы с текстами (на CIAN есть описания). Дома варим всё, что придумаем. На следущей паре делимся находками. 
	
	\item Ищем в данных по недвижимости аномалии. Думаем как поступить с аномалиями.  
	
	\item Ищем разные кластерные структуры. Размышляем о фроде и фродерах. 
	
	\item Визуализируем всё, что только можно. Смотрим на plotly. 
	
	\item Отрабатываем всё, что в первом семестре узнали на эконометрике. Оцениваем предельные эффекты, проверяем интересные гипотезы.  Дома проверяем ещё!
	
	\item Пробуем нонстопом на небольших субсэмплах построить кучу моделей. От линейных и соседей до деревьев и бустинга. Дома пытаемся улучшить результаты. 
	
	\item  Обсуждаем улучшения. Пытаемся улучшить вместе. Квантильная регрессия, блендинг и стекинг. 
	
	\item Flask. Создаём свою первую веб-страницу для презентации результатов.
	
	\item Снова наводим порядок в проекте. Доводим до конца всё, над чем начали работать. Начинаем соединять лучшие решения в одном репозитории.
\end{enumerate}


\end{document} 
